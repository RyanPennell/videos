%%%%%%%%%%%%%%%%%%%%%%%%%%%%%%%%%%%%%%%%%
% Beamer Presentation
% LaTeX Template
% Version 1.0 (10/11/12)
%
% This template has been downloaded from:
% http://www.LaTeXTemplates.com
%
% License:
% CC BY-NC-SA 3.0 (http://creativecommons.org/licenses/by-nc-sa/3.0/)
%
%%%%%%%%%%%%%%%%%%%%%%%%%%%%%%%%%%%%%%%%%

%----------------------------------------------------------------------------------------
%	PACKAGES AND THEMES
%----------------------------------------------------------------------------------------

\documentclass{beamer}

\mode<presentation> {

% The Beamer class comes with a number of default slide themes
% which change the colors and layouts of slides. Below this is a list
% of all the themes, uncomment each in turn to see what they look like.

\usetheme{default}
%\usetheme{AnnArbor}
%\usetheme{Antibes}
%\usetheme{Bergen}
%\usetheme{Berkeley}
%\usetheme{Berlin}
%\usetheme{Boadilla}
%\usetheme{CambridgeUS}
%\usetheme{Copenhagen}
%\usetheme{Darmstadt}
%\usetheme{Dresden}
%\usetheme{Frankfurt}
%\usetheme{Goettingen}
%\usetheme{Hannover}
%\usetheme{Ilmenau}
%\usetheme{JuanLesPins}
%\usetheme{Luebeck}
%\usetheme{Madrid}
%\usetheme{Malmoe}
%\usetheme{Marburg}
%\usetheme{Montpellier}
%\usetheme{PaloAlto}
%\usetheme{Pittsburgh}
%\usetheme{Rochester}
%\usetheme{Singapore}
%\usetheme{Szeged}
%\usetheme{Warsaw}

% As well as themes, the Beamer class has a number of color themes
% for any slide theme. Uncomment each of these in turn to see how it
% changes the colors of your current slide theme.

%\usecolortheme{albatross}
%\usecolortheme{beaver}
%\usecolortheme{beetle}
%\usecolortheme{crane}
%\usecolortheme{dolphin}
%\usecolortheme{dove}
%\usecolortheme{fly}
%\usecolortheme{lily}
%\usecolortheme{orchid}
%\usecolortheme{rose}
%\usecolortheme{seagull}
%\usecolortheme{seahorse}
%\usecolortheme{whale}
%\usecolortheme{wolverine}

%\setbeamertemplate{footline} % To remove the footer line in all slides uncomment this line
%\setbeamertemplate{footline}[page number] % To replace the footer line in all slides with a simple slide count uncomment this line

%\setbeamertemplate{navigation symbols}{} % To remove the navigation symbols from the bottom of all slides uncomment this line
}
\usepackage{amsmath}
\usepackage{graphicx} % Allows including images
\usepackage{booktabs} % Allows the use of \toprule, \midrule and \bottomrule in tables

%----------------------------------------------------------------------------------------
%	TITLE PAGE
%----------------------------------------------------------------------------------------

\title[Short title]{Introduction to Matrix Algebra\\ with\\ Python, Numpy, and Scipy} % The short title appears at the bottom of every slide, the full title is only on the title page

\author{Ryan Pennell} % Your name
\institute[Ivy Tech] % Your institution as it will appear on the bottom of every slide, may be shorthand to save space
{
Ivy Tech \\ % Your institution for the title page
\medskip
\textit{contact@ryanpennell.xyz} % Your email address
}
\date{\today} % Date, can be changed to a custom date

\begin{document}

\begin{frame}
\titlepage % Print the title page as the first slide
\end{frame}

\begin{frame}
\frametitle{Matrix Addition}

\[
\begin{bmatrix}
    x_{11}       & x_{12} & x_{13} & \dots & x_{1n} \\
    x_{21}       & x_{22} & x_{23} & \dots & x_{2n} \\
    \hdotsfor{5} \\
    x_{d1}       & x_{d2} & x_{d3} & \dots & x_{dn}
\end{bmatrix}
+
\begin{bmatrix}
    y_{11} & y_{12} & y_{13} & \dots  & y_{1n} \\
    y_{21} & y_{22} & y_{23} & \dots  & y_{2n} \\
    \vdots & \vdots & \vdots & \ddots & \vdots \\
    y_{d1} & y_{d2} & y_{d3} & \dots  & y_{dn}
\end{bmatrix}
\]

\[
=
\begin{bmatrix}
	x_{11}+ y_{11}  &  x_{12}+ y_{12} & x_{13}+ y_{13} & \dots & x_{1n}+y_{1n}\\
	x_{21}+y_{21}   & x_{22}+y_{22} & x_{23}+Y_{23} & \dots & x_{2n}+y_{2n} \\
	\vdots & \vdots & \vdots & \ddots & \vdots \\
	x_{d1}+y_{d1}  & x_{d2}+y_{d2} & x_{d3}+x_{d3} & \dots & x_{dn}+Y_{dn}
\end{bmatrix}
\]

\end{frame}

\begin{frame}
\section{Example}
Let
\frametitle{Addition Example}
\[A=
\begin{bmatrix}
	1 & 2 \\
	3 & 4\\
\end{bmatrix}
B=
\begin{bmatrix}
	4 & 3\\
	2 & 1\\
\end{bmatrix}
\]
Then 
\[
A+B=
\begin{bmatrix}
	1+4 & 2+3\\
	3+2 & 4+1 \\
\end{bmatrix}
=
\begin{bmatrix}
	5 & 5\\
	5 & 5\\	
\end{bmatrix}
\]
\end{frame}

\begin{frame}
\frametitle{Addition Example ctd.}
Notice 
\[
A+B=
\begin{bmatrix}
	1+4 & 2+3\\
	3+2 & 4+1 \\
\end{bmatrix}
= B + A 
\begin{bmatrix}
	4+1 & 2+3\\
	2+3 & 4+1\\	
\end{bmatrix}
=
\begin{bmatrix}
	5 & 5\\
	5 & 5\\	
\end{bmatrix}
\]
\begin{example}
$>>>$ A = np.matrix('1 2; 3 4')\\
$>>>$ B = np.matrix('4 3; 2 1')\\
$>>>$ print(A+B)\\
$>>>$ print(B+A)\\
\end{example}
$$A+B=B+A$$
Matrix addition and subtraction is commutative. However, the dimensions of the matrices must be equal. 
\end{frame}
\begin{frame}
\frametitle{General rules of addition for matrices}
Matrix addition and subtraction is commutative. However, the dimensions of the matrices must be equal. \\
General Rule: 
$$\boxed{A+B=\sum_{ij}(a_{ij}+b_{ij})_{ij}}$$
\end{frame}
\begin{frame}
\frametitle{Matrix multiplication}
General Rule:
$$(AB)_{ij}=\sum_{k=1}^n(a_{ik}b_{kj})$$
\end{frame}
\begin{frame}
\[A=
\begin{bmatrix}
	1 & 2 \\
	3 & 4\\
\end{bmatrix}
B=
\begin{bmatrix}
	4 & 3\\
	2 & 1\\
\end{bmatrix}
\]
Then 
\[
AB=
\begin{bmatrix}
	1\cdot 4 +2\cdot 2& 1\cdot 3+ 2\cdot 1\\
	3\cdot 4+4\cdot 2 & 3\cdot 4+4\cdot 1 \\
\end{bmatrix}
=
\begin{bmatrix}
	8& 5\\
	20 & 13\\	
\end{bmatrix}
\]
However


\[
BA=
\begin{bmatrix}
	4\cdot 1 +3\cdot 3& 4\cdot 2+ 3\cdot 4\\
	2\cdot 1+1\cdot 3 & 2\cdot 2+1\cdot 4 \\
\end{bmatrix}
=
\begin{bmatrix}
	13& 20\\
	5& 8\\	
\end{bmatrix}
\]
So, $$AB\neq BA$$
This means that matrix multiplication is not commutative. 
\end{frame}
\begin{frame}
\begin{example}[code]

$>>>$ print(A*B)\\
$>>>$ print(B*A)
\end{example}
\end{frame}
\begin{frame}
\frametitle{Dimensions}
If $A$ is $n\times m$ and $B$ is $n\times m$ then $A+B$ will be of dimensions $n\times m$. For matrix addition and subtraction the dimensions of $A$ and $B$ have to be the same. 
$$$$

For matrix multiplication the columns of the first matrix have to be equal to the rows of the second.  So, if $A$ has dimensions of $n\times m$ then $B$ dimensions of $m\times k$, $m\times n$ or $m\times m$.
\end{frame}

\begin{frame}
\frametitle{Identity and null}
Identity matrix is like the number one and can be as large or small as you need it. Always square. 
\[I=[1]=
\begin{bmatrix}
1 & 0 \\
0 & 1\\
\end{bmatrix}
=
\begin{bmatrix}
 1 & \dots & 0\\
0 & 1 & \dots\\
0 & \dots & 1\\
\end{bmatrix}
\]
Zero matrix (does not have to be square).
\[
\begin{bmatrix}
0 & 0 & 0 & 0
\end{bmatrix},
\begin{bmatrix}
0 & 0 & 0 & 0\\
0 & 0 & 0 & 0\\
\end{bmatrix},
\begin{bmatrix}
0 & 0 & 0 & 0\\
0 & 0 & 0 & 0\\
0 & 0 & 0 & 0\\
\end{bmatrix}
\]
\end{frame}
\begin{example}[code]
$>>>$ np.identity(1)\\
$>>>$ np.identity(2)\\
$>>>$ np.identity(3)\\
$>>>$ np.identity(1000000000)\\
$>>>$ MemoryError
$$$$
$>>>$ np.zeros((4))\\
$>>>$ np.zeros((2,4))\\
$>>>$ np.zeros((3,4))\\
$>>>$ np.zeros((4,4))\\


\end{example}
\begin{frame}
\frametitle{Dividing?}
Inverse of a matrix:
$$A\cdot A^{-1}=I=A^{-1}\cdot A$$
\end{frame}

\begin{frame}
\frametitle{example}
\[A=
\begin{bmatrix}
	1 & 2 \\
	3 & 4\\
\end{bmatrix}
\]
$$A\cdot A^{-1}=I$$
$$\implies A^{-1}=
\begin{bmatrix}
-2& 1\\
\frac 32 & -\frac 12 \\
\end{bmatrix}
$$
\begin{example}[code]
$>>>$from scipy import linalg\\
$>>>$linalg.inv(A)
\end{example}
\end{frame}
\begin{frame}
\frametitle{system of equations}
$$x+3y+5z=10$$
$$2x+5y+z=8$$
$$2x+3y+8z=3$$
Let 
$$A=
\begin{bmatrix}
1 & 3 &5\\
2 & 5 & 1\\
2 & 3 & 8\\
\end{bmatrix}
\text{and } B=
\begin{bmatrix}
10\\
8\\
3\\
\end{bmatrix}$$
Then $$
\begin{bmatrix}
x\\
y\\
z\\
\end{bmatrix}
=
\begin{bmatrix}
-9.28\\
5.16\\
0.76\\
\end{bmatrix}
$$
\end{frame}
\begin{frame}
\begin{example}[code]
$>>>$ A = np.matrix('1 3 5; 2 5 1; 2 3 8')\\
$>>>$ B = np.matrix('10; 8; 3')\\
$>>>$ np.linalg.solve(A, B)

\end{example}
\end{frame}

\begin{frame}
\frametitle{Transpose}
\[A=
\begin{bmatrix}
	1 & 2 \\
	3 & 4\\
\end{bmatrix}
,A^T=
\begin{bmatrix}
1 & 3\\
2 & 4\\
\end{bmatrix}
\]
\begin{example}[code]
$>>>$ np.transpose(A)
\end{example}
\end{frame}

\begin{frame}
\frametitle{More Adv. topics???}
\begin{description}

\item Determinants\\
Scalar Multiplication\\
Eigen-Values\\
Eigen-Vectors\\
Cross-Products\\
Markov-Chains\\
Stable-Vectors\\
\end{description}

\end{frame}

%------------------------------------------------


%----------------------------------------------------------------------------------------

\end{document} 
